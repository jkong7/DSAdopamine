% Hello!, you can use the percent symbol in order to write a commment in LaTeX.
% I will try my best to document my LaTeX here so you know what's going on


% These few lines are just to set up the LaTeX document by telling it what kind of document it is, as well as what packages/libraries I want to import and use. The ones here should be sufficient, but you should still import as needed.
\documentclass[12pt]{article}
\usepackage[margin=1in]{geometry}
\usepackage[all]{xy}

\usepackage{amsmath,amsthm,amssymb,color,latexsym}
\usepackage{mathtools}
\usepackage{geometry}        
\geometry{letterpaper}    
\usepackage{graphicx}
\usepackage{tabto}
\usepackage{setspace}

\usepackage[noframe]{showframe}
\usepackage{framed}

% Sets up "environments" of formatted text.
\renewenvironment{shaded}{%
  \def\FrameCommand{\fboxsep=\FrameSep \colorbox{shadecolor}}%
  \MakeFramed{\advance\hsize-\width \FrameRestore\FrameRestore}}%
 {\endMakeFramed}
\definecolor{shadecolor}{gray}{0.90}

\newtheorem{problem}{Problem}

\newenvironment{solution}[1][\it{Solution:}]{\textbf{#1 } }

% Sets spacing for the document
\onehalfspacing

% All latex documents begin with a `\begin{document}`
\begin{document}
\begin{titlepage}
   \begin{center}
       \vspace*{9cm}

       \textbf{CS 212 Winter 2025}

       \vspace{0.5cm}
        Homework \#3
            
        \vfill

       \textbf{Jonathan Kong}\\
       NetID dwa0713\\
       Collaborators: Jonathan Kong\\             
       Due Date: 30 Jan, 2015
            
   \end{center}
\end{titlepage}
%\noindent CS 212 Fall 2024\hfill Problem Set \# (insert number here)\\
%Name: Kevin Fan \hfill Due 14 Mar, 2015\\ 
%NetID: (Insert your NetID here) \\

%\hrulefill
\pagebreak

\begin{problem}
Divided Up 
\end{problem}

\begin{shaded}
\begin{solution}
\\ 
\noindent 
\textbf{a.} \fbox{False} \textbf{Disproof}: Consider \( a = 3, \, b = 6, \, c = 30 \). \( a \mid b \) and \( b \mid c \), but \( ab = 18 \nmid c \). \\
\noindent 
\textbf{b.} \fbox{True} \textbf{Proof:} Consider the contrapositive of the given statement:
\[
\neg (a \nmid b \lor a \nmid c) \to \neg (a \nmid bc)
\]
This is logically equivalent to:
\[
(a \mid b \land a \mid c) \to a \mid bc.
\]

Assume \(a \mid b\) and \(a \mid c\). By the definition of divisibility, there exists an integer \(k\) such that \(b = ak\), and there exists an integer \(j\) such that \(c = aj\). 

Multiplying these two expressions, we get:
\[
bc = (ak)(aj) = a^2(kj).
\]

Since \(a\), \(k\), and \(j\) are all integers, the product \(kj\) is also an integer by closure properties. Let \(z = kj\), where \(z\) is an integer. Thus:
\[
bc = a(z).
\] 

Therefore, \(a \mid bc\). \\
\noindent 
\textbf{c.} \fbox{True} \textbf{Proof:} By the rules of divisibility, there exists an integer \(k\) such that \(b = ak\), and there exists an integer \(j\) such that \(b + c = aj\). 

Subtracting the two equations, we have:
\[
c = (b + c) - b = aj - ak = a(j - k).
\]

Since \(j\) and \(k\) are both integers, their difference \(j - k\) is also an integer by closure properties. Let \(z = j - k\). Thus:
\[
c = a z.
\]

Therefore, \(a \mid c\). 
\end{solution}
\end{shaded}

% Creates a new page
\pagebreak


\begin{problem}
Oddly Divisible 
\end{problem}

\begin{shaded}
\begin{solution}
\\
\noindent 
\textbf{Proof:} Let \(k = 2n + 1\) for some integer \(n\). Substituting \(k\) into \(k^2 - 1\), we get:
\[
k^2 - 1 = (2n + 1)^2 - 1 = 4n^2 + 4n.
\]
Thus, the proof reduces to showing that \(8 \mid 4n^2 + 4n\).

Factor \(4n^2 + 4n\) as:
\[
4n^2 + 4n = 4 \cdot n(n + 1).
\]
Since \(n\) and \(n + 1\) are consecutive integers, one of them is always even. Therefore, \(n(n+1)\) is divisible by \(2\). Specifically, we can write:
\[
n(n+1) = 2b \quad \text{for some integer } b.
\]

Substituting \(n(n+1) = 2b\) into \(4 \cdot n(n+1)\), we get:
\[
4 \cdot n(n+1) = 4 \cdot 2b = 8b.
\]

To prove \(8 \mid 8b\), we need to show that there exists an integer \(z\) such that:
\[
8b = 8z.
\]
Let \(z = b\). Since \(b\) is an integer (as defined earlier when \(n(n+1) = 2b\)), \(z\) is also an integer by substitution. Thus:
\[
8b = 8z,
\]
proving that \(8 \mid 8b\).

Finally, since \(4n^2 + 4n = k^2 - 1\), it follows that:
\[
8 \mid k^2 - 1.
\]
\end{solution}
\end{shaded}


\pagebreak


\begin{problem}
Properties of Mod 
\end{problem}
\begin{shaded}

\begin{solution}
\\
\noindent 
\textbf{a.} \fbox{True} \textbf{Proof:} \(a \equiv b \pmod{m}\) yields that: 
\[
m \mid (a - b),
\]
or equivalently, there exists some integer \(k \) such that:
\[
a - b = mk.
\]

To prove \(ac \equiv bc \pmod{mc}\), we need to show that \(mc \mid (ac - bc)\). This is equivalent to proving that there exists some integer \(j \) such that:
\[
ac - bc = mcj.
\]

Multiply both sides of \(a - b = mk\) by \(c\):
\[
(ac - bc) = mck.
\]

Since \(k \in \mathbb{Z}\), it follows that \(mck = mcj\), where \(j = k \in \mathbb{Z}\). Thus:
\[
mc \mid (ac - bc).
\]

By the definition of congruence, this yields: 
\[
ac \equiv bc \pmod{mc}.
\]
\noindent 
\textbf{b.} \fbox{True} \textbf{Proof:} Prove using a biconditional proof. Specifically, we show:

\[
a \equiv b \pmod{12} \iff \left(a \equiv b \pmod{2} \land a \equiv b \pmod{6}\right)
\]

\textbf{First Direction:} (\(a \equiv b \pmod{12} \rightarrow a \equiv b \pmod{2} \text{ and } a \equiv b \pmod{6}\))

Assume \(a \equiv b \pmod{12}\). By the definition of congruence, this means:
\[
12 \mid (a - b),
\]
so there exists some integer $k$ such that:
\[
a - b = 12k.
\]

Since \(12 = 2 \cdot 6\), we can write:
\[
a - b = 2(6k),
\]
which shows that \(2 \mid (a - b)\), so \(a \equiv b \pmod{2}\).

Similarly, we can write:
\[
a - b = 6(2k),
\]
which shows that \(6 \mid (a - b)\), so \(a \equiv b \pmod{6}\).

Thus, \(a \equiv b \pmod{12} \rightarrow a \equiv b \pmod{2} \text{ and } a \equiv b \pmod{6}\).

\textbf{Second Direction:} (\(a \equiv b \pmod{2} \text{ and } a \equiv b \pmod{6} \implies a \equiv b \pmod{12}\))

Assume \(a \equiv b \pmod{2}\) and \(a \equiv b \pmod{6}\). By the definition of congruence, this means:
\[
2 \mid (a - b) \quad \text{and} \quad 6 \mid (a - b),
\]
so there exist integers $k$ and $j$ such that:
\[
a - b = 2k \quad \text{and} \quad a - b = 6j.
\]

Thus, \(a - b\) is a common multiple of \(2\) and \(6\). The least common multiple of \(2\) and \(6\) is \(12\), so \(a - b\) must be divisible by \(12\). Therefore, there exists some integer $z$ such that:
\[
a - b = 12z.
\]

This shows that \(12 \mid (a - b)\), which implies:
\[
a \equiv b \pmod{12}.
\]

Both directions have been proven, so:
\[
a \equiv b \pmod{12} \iff \left(a \equiv b \pmod{2} \text{ and } a \equiv b \pmod{6}\right).
\]

\end{solution}
\end{shaded}


\pagebreak


\begin{problem}
Mod Madness 
\end{problem}

\begin{shaded}
\begin{solution}
\\
\noindent 
Prove using a biconditional proof. Specifically, we show:
\[
n \text{ is even} \iff n^2 \equiv 0 \pmod{4}.
\]

\textbf{First Direction:} (\(n \text{ is even} \rightarrow n^2 \equiv 0 \pmod{4}\))

Assume \(n\) is even. Then, by definition, there exists some integer $k$ such that:
\[
n = 2k.
\]
Squaring both sides gives:
\[
n^2 = (2k)^2 = 4k^2.
\]
Since $k^2$ is an integer, there exists some integer $j$ such that \(n^2=4j\)\
This shows \(4 \mid n^2\), which implies:
\[
n^2 \equiv 0 \pmod{4}.
\]

\textbf{Second Direction:} (\(n^2 \equiv 0 \pmod{4} \rightarrow n \text{ is even}\))

Assume \(n^2 \equiv 0 \pmod{4}\). By the definition of congruence, this means:
\[
4 \mid n^2,
\]
so there exists some integer $k$ such that:
\[
n^2 = 4k.
\]
Rewriting, we have:
\[
n \cdot n = 2 \cdot 2k .
\]
Here, \(n\) must contribute a factor of 2 towards the \(4k\) product as \(n\) times itself gets the \(2 \cdot 2k\). Therefore, \(n = 2j\) for some integer $j$, which implies \(n\) is even.

Both directions have been proven, so:
\[
n \text{ is even} \iff n^2 \equiv 0 \pmod{4}.
\]

\end{solution}
\end{shaded}

\pagebreak


\begin{problem}
Goldbach's Conjecture
\end{problem}

\begin{shaded}
\begin{solution}
\\
\noindent 
\textbf{Proof:}  
We aim to prove that every odd integer \(n > 7\) can be expressed as the sum of three odd prime numbers, assuming Goldbach's Conjecture.

Let \(n\) be an odd integer greater than \(7\). We can write:
\[
n = 5 + (n - 5),
\]
where \(5\) is an odd prime and \(n - 5\) is even (since \(n\) and 5 are odd, $n-5$ must be even as the sum of two odd numbers is even). Since \(n > 7\), \(n - 5 > 2\), and Goldbach's Conjecture applies to the even number \(n - 5\). By Goldbach's Conjecture, we can write:
\[
n - 5 = p_1 + p_2,
\]
where \(p_1\) and \(p_2\) are primes (not necessarily distinct). Substituting back, we have:
\[
n = 5 + p_1 + p_2.
\]

Now, we consider the cases for \(p_1\):

\textbf{Case 1:} $p_1$ is even. The only prime even is 2 so $p_1$ must be 2. Similarly, since $n-5=2+p_2$, $p_2$ must be even since the difference of two even numbers is even. Therefore, $p_2$ is also 2. Substituting, we get:
\[
n = 5 + 2 + 2 = 9.
\]
For \(n = 9\), we verify that it can still be expressed as the sum of three odd primes:
\[
9 = 3 + 3 + 3.
\]
Thus, this special case holds.

\textbf{Case 2:} \(p_1\) is odd.  
If \(p_1\) is odd, then \(p_2 = n - 5 - p_1\) must also be odd, since \(n - 5 \) is even and the difference between an even integer and an odd integer is odd. Thus, both \(p_1\) and \(p_2\) are odd primes. Adding \(5\) (which is also an odd prime), we have:
\[
n = 5 + p_1 + p_2,
\]
expressed as the sum of three odd primes.

In both cases:
\begin{enumerate}
    \item If \(p_1 = 2\) and \(p_2 = 2\), we verify \(n = 9\) works as \(9 = 3 + 3 + 3\).
    \item If \(p_1\) and \(p_2\) are odd primes, \(n = 5 + p_1 + p_2\) is the sum of three odd primes.
\end{enumerate}
Thus, every odd integer \(n > 7\) can be expressed as the sum of three odd prime numbers. 

\end{solution}
\end{shaded}

\pagebreak

\begin{problem}
Bonus: Pythagorean Triples 
\end{problem}

\begin{shaded}
\begin{solution}\\
\noindent
Consider the contrapositive of the given statement: 
\[
\text{If } a^2 + b^2 = c^2, \text{ then } a \text{ or } b \text{ is even}.
\]

This is logically equivalent to:  
\[
\text{If } a \text{ is odd and } b \text{ is odd, then } a^2 + b^2 \neq c^2.
\]
Assume \(a\) and \(b\) are both odd. Then we can write:
\[
a = 2k + 1 \quad \text{and} \quad b = 2j + 1,
\]
for some integer \(k\) and integer \(j\). Plugging these into \(a^2 + b^2\), we get:
\[
a^2 + b^2 = 2(2k^2 + 2k + 2j^2 + 2j + 1).
\]
Since \(k\) and \(j\) are integers, by closure properties of integers, \(2k^2 + 2k + 2j^2 + 2j + 1\) is an integer. Let:
\[
z = 2k^2 + 2k + 2j^2 + 2j + 1,
\]
so:
\[
a^2 + b^2 = 2(2z + 1).
\]

The expression \(2(2z + 1)\) is the product of an even integer \(2\) and an odd integer \((2z + 1)\). This product cannot be expressed as the square of any integer, because a square must either be the product of two even integers or two odd integers. Therefore:
\[
a^2 + b^2 \neq c^2.
\]

This proves the contrapositive and thus establishes the original statement:  
\[
\text{If } a^2 + b^2 = c^2, \text{ then } a \text{ or } b \text{ is even}.
\]
\end{solution}
\end{shaded}

\pagebreak
\begin{problem}
Feedback
\end{problem}

\begin{shaded}
\begin{solution}\\
\noindent
\centerline{\fbox{\textbf{Three} hours + \textbf{25} minutes for the bonus problem}}
\end{solution}
\end{shaded}

%%%%%%%%%%%%%%%%%%%%%%%%%%%%%%%%%%%%%%%%%%%%%%%%%%%%%%%%%%
%%%%% Continue with this pattern if there are more %%%%%%%
%%%%%%%%%%%%%%%%% homework problems %%%%%%%%%%%%%%%%%%%%%%
%%%%%%%%%%%%%%%%%%%%%%%%%%%%%%%%%%%%%%%%%%%%%%%%%%%%%%%%%%
 
\end{document}
